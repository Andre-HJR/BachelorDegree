
\subsection{异步事件的底层依赖: Event}

需要设计一个事件监听的工具,使得可以将同步的数据结构转化为异步的数据结果,为异步提供基础的依赖。

异步事件的内部同步的数据结构为:

\begin{lstlisting}[caption=异步事件底层的同步结构]
struct Inner {
    notified: AtomicUsize,
    list: Mutex<List>,
    cache: UnsafeCell<Entry>,
}
\end{lstlisting}

Inner中的notified用于通知已被通知的Entry的数目,如果所有条目都被通知了,或者没有条目被通知,该值都会被设置为usize::MAX。而list则是用于指向已被注册的监听的链表, 而Mutex提供在多线程中list资源的锁机制,以解决资源竞争带来的数据冲突。

基于设计的底层同步结构,事件库将提供事件原语和状态机制,以此为系统上层异步改造提供基础组件。

\subsubsection{异步事件的一些原语}
\label{sssec:event}

\begin{lstlisting}[caption = 监听者的注册]
pub fn listen(&self) -> EventListener {
    let inner = self.inner();
    let listener = EventListener {
        inner: unsafe { Arc::clone(&ManuallyDrop::new(Arc::from_raw(inner))) },
        entry: Some(inner.lock().insert(inner.cache_ptr())),
    };

    // Make sure the listener is registered before whatever happens next.
    full_fence();
    listener
}
\end{lstlisting}

监听者的创建, full\_fence 该函数会阻止编译器和CPU围绕对某些类型的内春操作重新排序。使其可以在一些原子操作中创建同步关系。


\subsubsection{Event异步事件的状态}
\begin{lstlisting}[caption=Event的状态]
enum State {
    /// 刚刚被创建
    Created,
    /// 已经接收到一个通知
    ///
    /// 如果这是个 `additional` 通知,`bool` 将为 `true`
    Notified(bool),
    /// 正在被一个异步任务 `poll`,保存了任务的 `waker`
    Polling(Waker),
}
\end{lstlisting}

只用当状态为`State::Notified(bool)`的时候,事件才是被通知, 因此有如下的事件监听机制:

\begin{lstlisting}[caption=Event的事件获取并判断事件是否完成]
impl Future for EventListener {
    fn poll(mut self: Pin<&mut Self>, cx: &mut Context<'_>) -> Poll<Self::Output> {
        let mut list = self.inner.lock();

        let entry = match self.entry {
            None => unreachable!("cannot poll a completed `EventListener` future"),
            Some(entry) => entry,
        };
        // ...
    }
}
\end{lstlisting}

如若事件没有完成,则抛出异常。


\begin{lstlisting}[caption=Event的事件Poll机制]
impl Future for EventListener {
    fn poll(mut self: Pin<&mut Self>, cx: &mut Context<'_>) -> Poll<Self::Output> {
        // ...
        let state = unsafe { &entry.as_ref().state };
        match state.replace(State::Notified(false)) {
            State::Notified(_) => {
                list.remove(entry, self.inner.cache_ptr());
                drop(list);
                self.entry = None;
                return Poll::Ready(());
            }
            State::Created => {
                state.set(State::Polling(cx.waker().clone()));
            }
            State::Polling(w) => {
                if w.will_wake(cx.waker()) {
                    state.set(State::Polling(w));
                } else {
                    state.set(State::Polling(cx.waker().clone()));
                }
            }
        }
        Poll::Pending
    }
}
\end{lstlisting}


当获取事件的状态的可变引用时, 当章台为Polling和Created的时候,事件将会被加入到注册的waker序列中,当为Notified时,事件将会从序列中移除,然后释放相应的资源,并返回Ready告知任务已经完成,其余状态统一返回Pending。

在上述的代码中,基于Future的trait接口,Output将会被返回一个异步的值。 只有当poll返回的值Ready()时,函数将完成的值进行Poll的封装,返回其状态完成的变体。否则,将返回其其他状态的变体,向其范围内全局的调用者发出该工作尚未完成的信号。

poll方法接收两个参数: Pin(固定地址空间)和Context(函数上下文)。前者的行为类似普通引用,只是将值固定到内存的位置。后者存在的目的时为了Waker的实例可以被异步任务接收。这使得异步任务可以发出信号,表面其此时的状态(完成或一部分已完成)。由于主任务知道其将在准备就绪时收到通知,因此不需要一遍一遍调用。