\chapter{总结}
\label{chap:Summary}

\section{异步锁的思路}

\begin{lstlisting}[caption=异步锁结构]
pub struct AsyncMutex<T: ?Sized> {
    state: AtomicUsize,
    lock_ops: Event,
    data: UnsafeCell<T>,
}
\end{lstlisting}

其中 AsyncMutex 的底层依赖是 Event 。

\begin{lstlisting}[caption=异步锁的工作机制, numbers=left, label=asyncmutex]
async fn acquire_slow(&self){
    loop {
        let listener = self.lock_ops.listen();

        match self
            .state
            .compare_exchange(0, 1, Ordering::Acquire, Ordering::Acquire)
            .unwrap_or_else(|x| x)
        {
            0 => return,

            1 => {}

            _ => break,
        }

        listener.await;
    }
    // ...
}
\end{lstlisting}

通过Event::listen()创建事件监听,从代码的第2行到第15行,如果异步锁没有被任何任务持有,则第一个尝试获取锁,当compare\_exchange的返回值为0时,则表示事件成功获取锁,代码第16行,listener.await则是等待锁的释放。 

于是  
\begin{lstlisting}[caption = 异步锁]
pub async fn lock(&self) -> AsyncMutexGuard<'_, T> {
        if let Some(guard) = self.try_lock() {
            return guard;
        }
        self.acquire_slow().await;
        AsyncMutexGuard(self)
    }
\end{lstlisting}

异步的lock函数返回一个异步的锁结构,依赖于Rust的生命周期,其会在生命周期结束的时候被释放。


\section{文件系统的异步改造}

\subsection{如何泛化异步接口}

Rust 的编译器默认并不支持 async trait function。 编译器会推荐使用 async-trait,但是经过其改装的 async trait并不是零开销的。其会将


\begin{figure}[htbp]
    \figureCapSet
	\centering
	\begin{minipage}{0.49\linewidth}%表示图片的占用那一列的宽度
		\centering
        \begin{lstlisting}[frame=none]
#[async_trait]
pub trait AsyncBlockDevice {
    async fn read(&self, block_id: usize, buf: &mut [u8]);
    async fn write(&self, block_id: usize, buf: &[u8]);
}
        \end{lstlisting}
	\end{minipage}
    \hfill
	%\hfill表示横向排,两张图片会自动保证一定的距离
    %\vfill表示自动排版,两张图片会自动保证一定的距离
    %或者直接在这里加空格来增加图片之间的距离
	\begin{minipage}{0.49\linewidth}
		\centering
        \begin{lstlisting}[frame=none]
pub trait AsyncBlockDevice {
    fn read(&self, block_id: usize, buf: &mut [u8]) -> Box<dyn Future<Output = Option<&[u8], &[u8])>>>;
    fn write(&self, block_id: usize, buf: &[u8]) -> Box<dyn Future<Output = Option<&[u8], &[u8])>>>;
}
        \end{lstlisting}
	\end{minipage}
    \caption{async trait 对异步接口的处理}
\end{figure}


这里就有两层开销了:

\begin{itemize}
    \item 动态调度的开销 dyn Future。这意味着所有异步驱动设备的 read和write 函数都比较难做一些编译器的优化。
    \item 内存分配的开销 Box。在键值存储里,read和write应该是一个会被经常调用的函数。trait 被 async-trait 改写成新的形式之后,每次调用read和write都需要在堆上新建一个对象。这会对程序的性能造成比较大的影响。
\end{itemize}
其实,Rust本身提供了泛型编程的思路,直接拿出一个可供参考的解答,对应上部分可以写出如下代码:


\begin{lstlisting}[caption = AsyncBlockDevice的泛型异步接口]
pub trait AsyncBlockDevice {
    type NextFuture<'a> = impl Future<Output = Option<(&'a [u8], &'a [u8])>>;

    fn read(&self, block_id: usize, buf: &mut [u8]) -> Self::NextFuture<'_>;
    fn write(&self, block_id: usize, buf: &[u8]) -> Self::NextFuture<'_>;
}
\end{lstlisting}


解释:

\begin{itemize}
    \item NextFuture 是由 read(或者 write)函数返回的,而正常实现的read(或者 write)函数,只能返回和\&self生命周期相同的引用。在NextFuture上标注生命周期,是为了保障Self 的生命周期至少和 NextFuture 的一样长。
    \item read 和 write 的返回值,从代码中可以知道,两个函数的返回值是一个 Future,并不是一个常见的异步函数的返回值,因此需要使用async move返回一个闭包以满足异步函数的返回约束。
    \item 既然 read 和 write 返回的是一个Future,而Rust函数是无法在其编译器中被识别为一个具体的类型的,因此,此处的NextFuture需要让编译器自动推导Feature的具体类型,通过 impl 告知编译器NextFuture相关函数的返回需要产生一个	Future。
\end{itemize}

\section{高阶进程通信}


\subsection{Mach IPC}


在Mach\pagescite{mach0}的微内核设计上, IPC是核心,而且是内核中最重要的组件。不同于支持IPC机制的操作系统,Mach提供了一个支持大多数操作系统的IPC机制。

Mach IPC机制,将进程之间的通行做了更一层的抽象,将通行机制的基础定为消息传递。而消息的数据量可大可小,小到几个字节,大到整个地址空间。内核需要确保进行数据传输时无需进行不必要的数据复制,同时其应当提供安全可靠的的通信,当且仅当线程获取授权时,才能发送和接收消息。

通信和内存管理时紧密相连的。IPC依赖于内存的写时复制进行大量数据的有效传递。

IPC支持用户任务之间,以及用户和内核之间的通信,且适用于那些基于客户端-服务器模型的应用程序。

可以将Mach IPC的设计思路平移到starfish中,为其可能拥有的网络服务提供更加健壮的底层通行依赖。