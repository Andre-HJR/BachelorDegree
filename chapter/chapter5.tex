\chapter{总结}
\label{chap:Summary}

得益于Rust的发展,使得操作系统的设计除了C语言之外又有了新的选择。 RISCV也是比较新颖的一种指令集架构,其更加简洁的汇编指令为探索和设计提供便利。

毕设所写的操作系统是一种尝试在调度器方面提供一种异步方案的解决思路,将内核与用户的调用信息在一个固定的内存空间段得以共享,这使得内核和用户都在一定程度上可以互相“看到”对方。从理论上来说这种设计存在缺陷,共享调度信息,使得内核的调度会暴露给用户,对系统运行埋下安全隐患。但得益于Riscv的特权级设计能很好地隔离不同的机器模式,因此可以确保内核运行的安全。

这一次毕设重新学习了一下操作系统,把以前不是特别牢固的知识重新做了一次梳理,学习了新的指令集(Riscv)和Rust的异步编程,填补了自己在操作系统学习上的不足,使得自己对操作系统开发有了更加深刻的认识,为以后的工作积累了经验。

由于自己能力和精力有限,系统设计可能存在些许不合理的地方。例如内核与用户之间的任务仍然是由riscv的指令集所隔离,在用户空间暴露内核的运行机制以及对于异步轮询的机制没有提供合适的解决方案等等。诸多不足,寄望后来人。
