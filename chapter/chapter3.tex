\chapter{系统设计}
\label{chap:SystemDesign}


\section{系统内核}

\subsection{内存管理}

\subsection{进程通信}

\subsubsection{哲学家就餐}

\section{异步I/O}

\subsection{ring\_scheduler}

\subsection{异步}

\section{模块设计}

\subsection{静态模块}

在动态模块的设计中,系统模块设计应当遵循以下原则:

\begin{enumerate}
    \item 最小正交化的模块
    \item 树状依赖和空间隔离
\end{enumerate}

\subsubsection{最小正交的模块}




\subsubsection{树状依赖和空间隔离}


\subsection{动态模块}

在动态模块的设计中,系统的模块设计应当遵循以下三个原则: \begin{enumerate}
    \item 需要获取所有模块的运行时的持久边界
    \item 最大限度地发挥语言(Rust)和编译器的作用
    \item 最小化模块之间的状态溢出
\end{enumerate}

\subsubsection{需要获取所有模块的运行时的持久边界}

系统内核中的模块组件具有明确定义的边界(严格化的正交模块),并在整个运行时保持不变: 在实现时,系统模块组件以Rust独立的Crate的形式存在;在编译时,系统模块组件以一组加载的内存区域的形式存在;在运行时,系统模块组件以一组加载内存区域的形式存在,内存区域具有每部分的边界和依赖元数据。

上述的设计原则每一个内核的系统模块组件都需要遵循。运行时,可以显示识别系统模块组件的边界是系统内核中组件隔离和状态管理的基础。

在运行时,系统内核根据需要将所有系统模块组件加载并链接到系统中。简而言之,这需要找到并解析系统模块组件对象,将其部分加载到内存中,解析其依赖,根据依赖树,以写入连接器重定位条目,根据需要递归加载任何丢失的系统模块组件,并向符号映射添加新的公共符号。基于此可以为内核进化和故障恢复提供理论基础。加载的系统模块组件集定义了一个系统模块组件空间,一个包含所有系统模块组件公共符号的真正的名称空间,用于快速解析单元格之间的依赖关系。每个加载的系统模块组件节点跟踪其组成部分和存储区域包含它们。每个系统模块组件中的部分对应于其Crate的目标文件中的部分,例如,可执行文件、只读数据和读写数据部分。每个加载的分段节点跟踪其大小、在存储器中的位置以及双向依赖性(输入和输出);额外的元数据用于加速系统模块组件交换和其他系统功能。

系统模块组件边界的持久性降低了复杂性: 系统内核的持久性系统模块组件边界在其存在的所有阶段提供了一致的系统结构抽象。这将会降低了开发者对系统的理想模型的复杂性,并简化了故障恢复和演化逻辑,因为系统内核可以在运行时从相同的面向系统模块组件的角度自省和管理它自己的代码。从顶层应用程序和库到核心内核组件的一切都可以作为系统模块组件来观察。这使得系统内核能够

\begin{itemize}
	\item 实现统一适用于任何单元的单一机制,即模块交换,以及
	\item 以安全的方式从多个系统层(例如,应用和内核组件)联合进化模块。
\end{itemize}

\subsubsection{最大限度地发挥语言(Rust)和编译器的作用}
通过使编译器能够最大限度地检查安全性和正确性不变量来最大限度发挥语言的力量。

将系统内核的执行环境与该语言的运行时模型相匹配,并在Rust等现代语言提供的强大的静态类型系统中实现操作系统概念。这将编译器检查的不变量(例如,没有悬空引用)扩展到了所有类型的资源,而不仅仅是语言中内置的那些。

依赖语言设计有两个主要好处:

\begin{itemize}
    \item 首先,它使编译器能够接管资源管理职责,减少了操作系统必须维护的状态,从而减少了状态溢出并加强了隔离。
    \item 它使编译器能够在理解代码行为的过程中应用安全检查,从应用程序到核心内核组件实现端到端安全,并将语义运行时错误转化为编译时错误。
\end{itemize}

相比之下,传统的非语言方法依赖于硬件保护和运行时检查来维护安全性、隔离性和正确性的不变量。这些特性对编译器是透明的,需要不安全的代码。甚至现有的安全语言操作系统。在语言级别的安全代码和底层的不安全核心之间有一个缺口,后者将语言所需的抽象实现为一个黑盒。

\subsubsection{最小化模块之间的状态溢出}

由于系统内核的组件结构是模块化的,因此状态溢出只能发生在跨越模块边界并导致接收单元状态改变的交互(例如,函数调用)中。
