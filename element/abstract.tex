% 中文摘要
\setlength{\headheight}{1.5cm}

\abstractcn


操作系统内核有几个发展节点, 一开始以裸机应用、批处理系统为主,后来逐渐出现多任务的操作系统,为了应对系统发展的需求,基于线程的系统应运而生;为了提升它的性能,有一些语言和编程架构,在应用层复用线程资源,提出了“协程”的概念,节省任务调度的开销。毕设工作尝试使得操作系统中不同资源被调度器所共享,调度器将所使用的资源同时共享给内核和用户,由此两者将拥有近乎相同的调度处理机制。同时,为了降低系统的耦合程度,尝试模块化系统内核,尝试将系统组件剥离。所以,工作如下:

\begin{itemize}
\item Ring\_scheduler: 自己的工作是继续tonado-os的工作, 实现一个类似 shared\_ scheduler 的ring\_scheduler,将其接入rCore并使得其可以工作在rCore中,并试着将其从内核模块中正交出来,形成一个独立的模块(Crate)。 

\item 模块化的设计:模块化思路,主要依赖于语言的编译时,其实质是将杂糅在一起的模块通过正交,形成独立且具有层次的的依赖树,模块之间的设计依赖通过各自之间API的调用形成约束。
\end{itemize}


毕设的目标是尝试利用Rust语言所提供的并发模型和模块管理,在内核空间中实现并发安全、模块化。而Rust的异步机制,则为操作系统并发提供了优化参考。模块化系统内核,为内核灵活开发提供帮助。

% 中文关键词
\keywordscn\quad 操作系统;Rust;异步;模块化;IO
% 英文摘要
\abstracten

The operating system kernel has several development nodes, at the beginning bare-metal applications and batch systems were the main focus, and then gradually emerged multitasking operating systems, and in response to the needs of system development, thread-based systems came into being In order to improve its performance, there are languages and programming architectures that reuse thread resources at the application level and propose In order to improve its performance, some languages and programming architectures reuse thread resources at the application level and propose the concept of "concurrent threads" to save the overhead of task scheduling. Bishop's work tries to make the different resources in the operating system shared by the scheduler, and the scheduler can save the overhead of scheduling. resources in the operating system are shared by the scheduler, and the scheduler shares the used resources with both the kernel and the user, so that both will The scheduler shares the used resources with both the kernel and the user, so that both will have nearly identical scheduling processing mechanisms. At the same time, in order to reduce the coupling of the system, an attempt was made to modularize the system kernel. to reduce the system coupling and try to modularize the system kernel and try to strip the system components. So, it works as follows:

\begin{itemize}
\item Ring\_scheduler: Its own work is to continue the work of tronado-os and implement a ring\_scheduler similar to shared\_
scheduler, plug it into rCore and make it work in rCore, and try to remove it from the kernel module.
and try to orthogonalize it out of the kernel module to form a separate module (Crate).

\item Modular design: The modular idea, which relies heavily on the compile time of the language, is essentially to form independent and hierarchical dependency trees by orthogonalizing modules together, with the design dependencies between modules bounded by API calls between them.
\end{itemize}


The goal of Bishop is to try to implement concurrency in kernel space in a safe and modular way using the concurrency model and module management provided by the Rust language. The asynchronous mechanism of Rust, on the other hand, provides an optimization reference for operating system concurrency. Modularized system kernel for flexible kernel development.

% 英文关键词
\keywordsen\quad OS;Rust;Asynchronous;Modularity;IO